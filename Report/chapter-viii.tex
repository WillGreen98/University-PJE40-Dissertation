\chapter{Future Work}

This chapter focuses on possible amendments for this project, be it design or structural alterations for potential ideas to be constructed. The development carried out throughout this project has seen breakpoints which have led to new implementational ideas based on this project’s scope, entirely novel models which are a result of research (discussed in chapter 2). This chapter breaks down those ideas into their respected backgrounds and outlines future work and life for this project.

\section{Derived from Literature Review} \label{section:DerivedfromLiteratureReview}

As discussed in section \ref{section:IdentifyingResearchGapsandIncludingNovelty}, gaps can be identified within the research conducted, this section will focus on furthering the statements to explain how the literature review exposed less explored theoretical concepts and their counterpart implementation.

\begin{itemize}
    \item (Adaptation) --- \textbf{\textit{Compare POS-Tagging and Enhanced versions of Word2Vec with Machine Learning Implementations for Sentiment Analysis and Text Classification}}. Previous comparative studies focus on traditional versions of Word2Vec implementations, whereby it is often to see similar group-set of NLP and ML techniques being used in the results, this is a small change but could have a big impact for comparative studies and domain analysis. As ML concepts are now able to make use of more computational power, it is possible to phase out the testing of traditional techniques in favour for their ML adaptation (fundamentals do not change).
    \item (Novel Implementation) --- \textbf{\textit{Investigate the use of Machine Learning with Support-Vector Machines (SVM) and use Neural Networks or Deep Learning for Ensemble Learning applied Sentiment Analysis on Student Feedback.}}
\end{itemize}

\section{Derived from Artefact}

N.Y.C

\section{Derived from Observations}

When developing and training this project’s artefact, ethics were taken into consideration and it was decided to only make use of open-source (predefined) datasets, this decision limited the search for useable data and as a result, this model made use of two datasets from higher academic institutions. It would be beneficial to have access to more data or datasets for higher accuracy when training.

Expanding the model to an outside host, this project was developed within an isolated environment (the developer’s personal environment) as the model was easier to contain and maintain external variables. This resulted in limiting the nature of the model and its potential scope as resources were limited, however, developing in this state did minimise risk and lower the fault tolerance of the text-classification model. It would be beneficial to containerise this model and execute training on a more capable machine such as a higher core server.

The original intention for this project included the planning, design, and development of multiple NLP machine-learning models to put against each other to see how different implementations of theory may affect performance when trained on different datasets and NLP domains, however, due to unforeseen circumstances, this project experienced several difficulties with management and overall development. If this project were to have further development, it would be within reason to explore deeper theoretical combinations as discussed in section 8.1. The project’s scope and limitations would not differ as there is a pretrained and predefined model to use as a reference point.
