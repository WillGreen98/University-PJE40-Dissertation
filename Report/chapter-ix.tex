\chapter{Conclusion}

% Recap what you did. In about one paragraph recap what your research question was and how you tackled it.
% Highlight the big accomplishments. Spend another paragraph explaining the highlights of your results. These are the main results you want the reader to remember after they put down the paper, so ignore any small details.
% Conclude. Finally, finish off with a sentence or two that wraps up your paper. I find this can often be the hardest part to write. You want the paper to feel finished after they read these. One way to do this, is to try and tie your research to the “real world.” Can you somehow relate how your research is important outside of academia? Or, if your results leave you with a big question, finish with that. Put it out there for the reader to think about to.
% Optional Before you conclude, if you don’t have a future work section, put in a paragraph detailing the questions you think arise from the work and where you think researchers need to be looking next.

Even though the modern approach was not implemented, the mathematical foundations it is built on suggests it is a viable model as an alternative amalgamation with a potential accuracy of 96\% and a loss value of 40.65.

The developer intends to continue their research in this project and implement different models for the domain.